\documentclass[12pt,preprint]{aastex}
\usepackage[margin=1in]{geometry}  
\usepackage{graphicx}
\usepackage{amssymb}


\def\Mpc{\mathrm{Mpc}}
\def\pc{\mathrm{pc}}
\def\Rsol{\mathrm{R_\odot}}
\def\Lsol{\mathrm{L_\odot}}

\renewcommand{\baselinestretch}{1.1}
\newcommand{\bqu}{\begin{quote}}
	\newcommand{\equ}{\end{quote}}
\newcommand{\br}{\langle}
\newcommand{\ke}{\rangle}
\def\f#1#2{\frac{#1}{#2}}

\def\ni{\noindent}
\def\J{{\rm\,J}}
\def\W{{\rm\,W}}
\def\s{{\rm\,s}}
\def\erg{{\rm\,erg}}
\def\cm{{\rm\,cm}}
\def\m{{\rm\,m}}
\def\km{{\rm\,km}}
\def\kg{{\rm\,kg}}
\def\mm{{\rm\,mm}}
\def\gm{{\rm\,g}}
\def\g{{\rm\,g}}
\def\d{\rm\,d}
\def\h{\rm\,h}
\def\mum{\,\mu{\rm m}}
\def\K{{\rm\,K}}
\def\yr{{\rm\,yr}}
\def\Hz{{\rm\,Hz}}
\def\days{{\rm\,days}}
\def\cals{{\rm\,cals}}
\def\mole{{\rm\,mole}}
\def\gal{{\rm\,gal}}
\def\calo{\,{\rm calorie}}
\def\dyne{\,{\rm dyne}}
\def\s{\,{\rm s}}
\def\at{\, {\rm atmosphere}}
\def\J{\,{\rm joule}}
\def\pomega{\tilde{\omega}}
\def\vs{\vspace{0.1in}}
\def\n{\vspace{0.1in} \ni }


\title{Problem Set 8}
\begin{document}
\maketitle
\centerline{Astro C161} 

\centerline{Due Friday, April 8, 4:00 pm}

\begin{enumerate}
\setcounter{enumi}{-1}

\item \textbf{TALC} \textit{(5 pts)}: Come your TALC section. There will be a sign-up sheet to record your attendance.

\item \textbf{The Power Spectrum} \textit{(40 pts)}: Answer part (b) in an essay form. It should answer all the following questions and be clear, concise, and orderly in doing so. 
	
Here is the angular power spectrum of our universe measuring the temperature fluctuations in the CMB:

\includegraphics[width=6in]{AngularPowerSpectrum}

\begin{enumerate}
	
	\item What physical scale does the highest perturbation peak correspond to? (Check out Ned's Cosmology Calculator \verb+http://www.astro.ucla.edu/~wright/CosmoCalc.html+) 

\item How does this relate to the matter power spectrum and the density perturbations? What do does the x-axis mean, relating the multipole moment $l$ to k-modes? What do the first three peaks mean in a physical sense and what universal parameters can they reveal to us? What about the very high $l$ peaks? What would changing, say, curvature and baryon density do to the power spectrum and why? For more background, look at Wayne Hu's tutorials (\verb+\http://background.uchicago.edu/index.html+)
and Max Tegmark's clickable widget (\verb+http://space.mit.edu/home/tegmark/movies.html+).
\end{enumerate}

\item \textbf{Playing with Perturbations} \textit{(40 pts)}: The growth of matter perturbations can be written as:

$$ \ddot{\delta} + 2 \frac{\dot{a}}{a}\dot{\delta} = \frac{4\pi G}{c^2} \epsilon_m \delta$$

\begin{enumerate}

\item Consider a flat, matter-only universe. Use the Friedmann equation to show that $\frac{4\pi G}{c^2} \epsilon_m = \frac{3}{2} H_0^2 a^{-3} = \frac{2}{3t^2} $.

\item Show that $\delta \propto t^{2/3} $ or $t^{-1}$.

\item Change variables so that derivatives of $\delta$ refer to scale factor rather than time. using the convention that $\delta' = \frac{d \delta}{da}$, show that

$$ \delta'' + \frac{\ddot{a}}{\dot{a}^2}\delta' + \frac{2}{a}\delta' = \frac{3\epsilon_m}{2\epsilon_{tot}}\frac{\delta}{a^2}$$

\item \bqu \textit{Using the Friedmann equation in a radiation and matter universe, show that the previous equation becomes} \equ

$$ \frac{d^2\delta}{dy^2} +\frac{2+3y}{2y(1+y)}\frac{d\delta}{dy}- \frac{3}{2y(1+y)}\delta = 0 $$

\bqu \textit{where $y = \frac{\epsilon_m}{\epsilon_r} $. What are the growing and decaying solutions for $\delta$ in this universe in terms of $y$ in the radiation-dominated regime ($y<<1$)?} \equ

\item \bqu \textit{When a perturbation in this universe is outside the horizon, it grows as $\delta \propto y^2$ in the RD era. When it enters the horizon, it is necessarily a mixture of growing and decaying modes. Show that} \equ

$$ \delta(y) \approx 2y_e^2[(1+3y/2)ln(1/y_e)- ln(1/y)] $$

\bqu \textit{where $y_e = y(t_{entry})$.} \equ


\end{enumerate}

\item \textbf{Presentation Outline} \textit{(15 pts)}: Upload the outline of your presentation to bCourses. Lay out what you're going to talk about, how you'll do it, and who will be doing it. Please cite a couple of your major sources as well.

\end{enumerate}

\end{document}  
