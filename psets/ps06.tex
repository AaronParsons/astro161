\documentclass[12pt,preprint]{aastex}
\usepackage[margin=1in]{geometry}  
\usepackage{graphicx}
\usepackage{amssymb}
\usepackage{hyperref} % for \url, etc

\def\Mpc{\mathrm{Mpc}}
\def\pc{\mathrm{pc}}
\def\Rsol{\mathrm{R_\odot}}
\def\Lsol{\mathrm{L_\odot}}

\renewcommand{\baselinestretch}{1.1}
\newcommand{\bqu}{\begin{quote}}
	\newcommand{\equ}{\end{quote}}
\newcommand{\br}{\langle}
\newcommand{\ke}{\rangle}
\def\f#1#2{\frac{#1}{#2}}

\def\ni{\noindent}
\def\J{{\rm\,J}}
\def\W{{\rm\,W}}
\def\s{{\rm\,s}}
\def\erg{{\rm\,erg}}
\def\cm{{\rm\,cm}}
\def\m{{\rm\,m}}
\def\km{{\rm\,km}}
\def\kg{{\rm\,kg}}
\def\mm{{\rm\,mm}}
\def\gm{{\rm\,g}}
\def\g{{\rm\,g}}
\def\d{\rm\,d}
\def\h{\rm\,h}
\def\mum{\,\mu{\rm m}}
\def\K{{\rm\,K}}
\def\yr{{\rm\,yr}}
\def\Hz{{\rm\,Hz}}
\def\days{{\rm\,days}}
\def\cals{{\rm\,cals}}
\def\mole{{\rm\,mole}}
\def\gal{{\rm\,gal}}
\def\calo{\,{\rm calorie}}
\def\dyne{\,{\rm dyne}}
\def\s{\,{\rm s}}
\def\ms{\,{\rm ms}}
\def\at{\, {\rm atmosphere}}
\def\J{\,{\rm joule}}
\def\pomega{\tilde{\omega}}
\def\vs{\vspace{0.1in}}
\def\n{\vspace{0.1in} \ni }


\title{Problem Set 6}
\begin{document}

\maketitle
\centerline{Astro C161} 

\centerline{Due Friday, March 4, 4:00 pm}

\begin{enumerate}
\setcounter{enumi}{-1}

\item \textbf{TALC} \textit{(5 pts)}: Come your TALC section. There will be a sign-up sheet to record your attendance. 

\item \textbf{Computing Nucleosynthesis} \textit{(95 pts)}: Here we will code a simulation of a simple model of the beginning of nucleosynthesis. If you haven't done any computational physics, \textbf{it is ok}. We wrote this problem so that you could do it even as a first-time programer. Use this skeleton of a Python program at \url{https://github.com/AaronParsons/astro161/blob/master/code/ps06_example.py}
\normalsize as a starting point (even if you want to start from scratch, you should use the same naming conventions for functions). 

We are all most comfortable working in Python, so it will be easiest for you to get help if you work in that. If you need to download a distribution of Python, we suggest using the Anaconda distribution (\url{https://www.continuum.io/downloads}). If you get errors or do not know how to implement a feature or function, please check the Python documentation (\url{https://docs.python.org/3/}) or Stack Overflow. As a rule, we will not be troubleshooting your errors ourselves because, if we start to do that, we will never get done fixing them.

When turning in your homework, please staple your code to the write-up.
 
\begin{enumerate}
	\item Write functions returning the number density of neutrinos vs. temperature
 and cross section of neutrinos vs. temperature until freeze-out.
 
	\item Calculate the Hubble constant as a function of time and the temperature as a function of time. Write functions for both of those calculations.

	\item Calculate the collision rate $\Gamma$ as a function of time. Write a function for that calculation.
	
	\item Calculate $t_{freeze}$. You will need this for a future function.
	 
	\item Write a function describing the ratio of the neutron and proton number density for time before freeze-out. ($\sim 1\ms$ to $t_{freeze}$) 
	
	\item Calculate the neutron decay rate from freeze-out up to deuterium formation. Add this calculation to the neutron-to-proton ratio function from the previous part. 
	
	\item Calculate the deuterium/neutron ratio vs. time. Write a function for that calculation. 
	
	\item Calculate the number density of protons vs. time. (Don't forget neutron decay where applicable.) 
	
	\item Plot the collision rate vs. time
	
	\item Plot the neutron/proton ratio vs. time. 
	
	\item Plot the number density of protons, neutrons, and deuterium vs. time without neutron decay.
	
	\item Plot the number density of protons, neutrons, and deuterium vs. time with neutron decay. Describe the difference between this plot and the last one.
	
	\item Give us the previous plot with all the same parameters except $\sigma_{\nu}$ has changed from $\sim 10^{-43}$ to $10^{-45}$. Give $Y_{max}$ for this new case and compare to the previous one.
	
	\item Why does the problem of describing nucleosynthesis become much tougher when we try to go to later times and fuse $\rm{^3He}$ and $\rm{^4He}$? 

\end{enumerate}

\end{enumerate}

\end{document}  
