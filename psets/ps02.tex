\documentclass[12pt,preprint]{aastex}
\usepackage[margin=1in]{geometry}  
\usepackage{graphicx}
\usepackage{amssymb}


\def\Mpc{\mathrm{Mpc}}
\def\pc{\mathrm{pc}}
\def\Rsol{\mathrm{R_\odot}}
\def\Lsol{\mathrm{L_\odot}}

\title{Problem Set 2}
\begin{document}
\maketitle
\centerline{Astro C161} 

\centerline{Due Friday, February 5, 4:00 pm}

\begin{enumerate}
\setcounter{enumi}{-1}

\item \textbf{TALC} \textit{(5 pts)}: Come to one of the two TALC sections. There will be a worksheet to record your attendance.

\item \textbf{Explaining Expansion} \textit{(16 pts)}: Answer the following questions in words.
\begin{enumerate}
	\item How are the Friedmann equation, the fluid equation, the acceleration equation, and the equation of state dependent on each other? What does this imply for how you can solve the equations in tandem?  
	\item In detail, if you had a pressure/energy-density relationship for a two-component universe, how would you solve for redshift as a function of energy density.
\end{enumerate}

\item \textbf{Acceleration and Deceleration of the Universe} \textit{(30 pts)}: 
	\begin{enumerate}
	\item Derive an expression for the redshift at which the deceleration turned into acceleration in a universe consisting
only of matter and dark energy (a good approximation of the recent universe). Your expression should contain only fundamental cosmological parameters at the present day such as $\Omega_{0,m}, \Omega_{0,\Lambda}, H_0, k$, etc. 
	\item Compute the redshift of that transition for the currently favored cosmology: a flat universe which $\Omega_{0,m} = .31,\; \Omega_{0,\Lambda} = .69,\; H_0 = 68\;\mathrm{km/s/Mpc} $.
	\item Compute the redshift for when matter density and dark energy density were equal, which is to say, when $\epsilon_m = \epsilon_\Lambda $. How does this relate to what you found in (b)?
    \item Cosmic Microwave Background data indicate that $\Omega_{0,m}+\Omega_{0,\Lambda}$ is close to one. Supernova 
data indicate that the expansion of the universe is currently accelerating. What is the bound on $\Omega_{0,\Lambda}$ implied by these two statements?
	\end{enumerate}

\item \textbf{Working in a Static Universe} \textit{(17 pts)}: Consider Einstein's static universe, in which the attractive force of the matter density $\epsilon_m$ is exactly balanced by the repulsive force of the cosmological constant $\Lambda = 8 \pi G \epsilon_\Lambda/c^2$. Suppose stars convert some matter into radiation. Will the universe start to expand, contract, or remain the same? Justify your answer.

\pagebreak
\item \textbf{Redshift and Time} \textit{(32 pts)}: 
	\begin{enumerate}
	\item A light source in a flat, single-component universe has a redshift $z$ when observed at a time $t$. Assume that the equation of state for the component is $P=w\epsilon$. Show that the observed redshift changes at a rate
	
	$$ \frac{dz}{dt} = H_0(1+z) - H_0(1+z)^{3(1+w)/2} $$
	
	For what values of $w$ does the redshift decrease with time? for what values of $w$ does the redshift increase with time?
	\item Suppose this is a matter-only universe with Hubble constant $H_0 \approx70\; \mathrm{km/s/Mpc}$. You observe a galaxy at $z=1$. How long will you have to keep observing the galaxy to see its redshift change by $0.0001\%$? 
	\end{enumerate}

\end{enumerate}

\end{document}  
